\chapter{Úvod}

V~posledních několika dekádách došlo ke~změně v~potřebách vyhledávání.
Už se nepožaduje/nestačí pouze exaktní nebo přibližné vyhledání textu
v~textu. S~příchodem masivního ukládání obrázků, zvukových záznamů
a videí nelze používat \uv{klasické} postupy pro indexování a následné
vyhledávání, ale musí se používat podobnostní vyhledávání. Praktické
použití je např.~rozpoznávání řeči, rozpoznávání tváří, atd\ldots{}

Jednou z~metod pro indexování a vyhledávání v~multimediálních databázích
je \MIndex\cite{Novak:2009:MIE:1637863.1638184}\@. Autoři této
metody, která vychází z~jiné indexovací metody iDistance\cite{Jagadish:2005:IAB:1071610.1071612},
jsou Martin Novák a Michal Baťko z~Masarykovi univerzity Brno\@.
M-index podstatných způsobem vylepšuje výkon při vyhledávání, použitím
jiných metod rozdělení datové prostoru i samotného vyhledávání pomocí
základních metod range query a hledání K~nejbližších sousedů (KNN
search)\@.

Metrický index definuje \emph{univerzální mapovací schéma} z~obecného
metrického prostoru do reálných čísel\@. Co je nejdůležitější, toto
schéma zachovává blízkost dat, tedy mapuje podobné metrické prvky
do~blízkých čísel\@. Indexovací a vyhledávací mechanismus \MIndex
u~používá množinu referenčních prvků (pivotů) a optimálně využívá všechny
známé principy pro dělení, větvení a filtrování v~metrickém prostoru\@.

Charakter mapy \MIndex u~umožňuje použít dobře známé techniky jako
je \BPTree\cite{Cormen:2001:IA:580470} nebo distribuované
hašovací tabulky\@.

V~této práci detailně vysvětluji principy \MIndex u, jeho implementaci
v~jazyce Java a porovnávám výkonnost implementací \MIndex v~jazycích
Java, \CC{} a C\#. Předpokladem je, že na stejném stroji a na stejné
množině dat, implementace v~\CC{} zpracuje požadavky v~nejkratším
čase. Následována implementací v~C\# nebo v~Java. Podle mého názoru,
lze zrychlit běh programu v~jazyce Java tím, že se dodatečně povolí
agresivní optimalizace a \uv{zahřátím} JVM. Toto zkoumám na závěr této práce.

\section{Multimediální databáze}

Multimediální databáze je databázový systém, který umožňuje spravovat
multimediální data\@.~Multimediální data jsou nestrukturovaná data,
vyznačující se velkým objemem\@.~Typy a použití těchto databází
jsou různé, např.:


\paragraph{Autentizační Multimediální Databáze}

Skenování oční duhovky a vyhodnocování autentičnosti\@. Lze použít
i v~medicíně při diagnostikování nemocí


\paragraph{Balistická databáze}

Porovnávání otisků na nábojnicích z~místa činu a kohoutkem střelné
zbraně, např.~systém Drugfire\cite{drugfire}

\todo{Další multimediální databáze}



\subsection{Problémy spojené s~multimediálníma databázemi}

V~multimediální databázi narážíme na několik problémů odlišných od
,,klasických'' RDBMS, případně oproti \emph{key-value }databázím\cite{no-sql}:
\begin{itemize}
\item prokletí počtu dimenzí (\emph{curse of dimensionality})\cite{Bellman195706}
\item multimediální soubory (objekty) jsou velké
\item datové struktury jsou pro jeden typ multimédií --- zvukový záznam
má jiná kritéria než obraz
\item problém indexování takových objektů
\end{itemize}

\section{Metrický prostor}


\subsection{Předpoklady pro metrické vyhledávaní}

\emph{Metrický prostor }$\mathcal{M}$ je pár $\mathcal{M}=(\mathcal{D},d)$,
kde $\mathcal{D}$ je \emph{doména} prvků a $d$ je \emph{funkce vzdálenosti}
$d\;:\;\mathcal{D}\times\mathcal{D}\rightarrow\mathbb{R}$ splňující
následující požadavky pro všechny $o,p,q\in\mathcal{D}$:

\begin{align*}
d(o,p) & \geq0 &  & \textrm{(nezápornost)}\\
d(o,p) & =0\Longleftrightarrow o=p &  & \textrm{(identita)}\\
d(o,p) & =d(p,o) &  & \textrm{(symetrie)}\\
d(o,q) & \leq d(o,p)+d(p,q) &  & \textrm{(trojúhelníková nerovnost)}
\end{align*}



\subsection{Podobností hledávání}

Metrický prostor jako model podobnosti je typicky prohledáván podle
vzoru \emph{example-by-query} \textendash{} dotaz je vytvořen dle
\emph{prvku} $d\in D$ a daného \emph{omezení} na data, která mají
být vybrána z~indexované množiny$X\subseteq D$\@. Nejjednodušší
jsou dva typy dotazů: \emph{range~query}~$R(q,r)$\footnote{Range Query je podrobně popsán viz. \vref{sec:Range-Query}
}, který vybere všechny prvky $o\subseteq X$ ve vzdálenosti $r$~od~$q$
(např. $\left\{ o\subseteq X\mid d(q,o)\leq r\right\} $), a \emph{nearest
neighbors query} $kNN(q,k)$, který vrátí $k$~nejbližších prvků
od $q$~z~$X$\@.

\MIndex využívá několika struktur\@. V~následujících částí se budeme
zabývat popisem pravidel (algoritmů), které jsou zapotřebí k~utvoření
a správného používání \MIndex u.