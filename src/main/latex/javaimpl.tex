\chapter{\MIndex{} implementace v~jazyce Java}


\section{Jazyk Java, JVM}

Jazyk Java je silně typovaný jazyk. Tento jazyk není interpretovaný,
výsledkem překladu je \bytecode{}.\footnote{\bytecode{} jsou instrukce pro JVM. Více viz. \prettyref{sub:JIT}} Až tento \bytecode{} je interpretován v~tzv. Java Virtual Machine (JVM\nomenclature{JVM}{Java Virtual Machine}).
Díky JVM může kód přeložený na jedné platformě (ať už Windows nebo Unix a 32bit nebo 64bit) běžet na jiné. \footnote{\emph{\uv{Write once, run anywhere}}\cite{lindholm2013java}}

Hlavní klíčové vlastnosti, jak jazyka Java, JVM a JDK\nomenclature{JDK}{Java Development Kit}
jsou tyto: automatická alokace a dealokace paměti, objektovost, generika,
překlad \bytecode{} do nativních instrukcí procesoru, bohatá
knihovna tříd v~základním JDK\ldots{}

\subsection{Automatická správa paměti -- garbage collection}

V~jazyce Java neexistují ukazatele, pouze reference. V~jazycích C/\CC{}
se musí vývojář starat o~správu paměti pomocí funkcí \method{malloc},
\method{free} (jazyk C a \CC), resp. operátorů \method{new} a \method{delete}
(jazyk \CC). V~jazyce Java, resp. v~JVM \cite{lindholm2013java} tuto starost přebírá tzv. \emph{garbage
collector} (GC\nomenclature{GC}{garbage collector}), který sleduje
dosažitelnost alokovaných objektů v~paměti JVM. Pokud GC
zjistí, že objekt není dosažitelný z~jiných objektů, uvolní pamět daného objektu.
\footnote{Podobná technika/pattern je součástí \CC 11 pomocí \type{std::shared\_ptr},
kde se používá \emph{reference counting} \cite{ISO:2012:CPP}. Tato technika není použitelná v~případě cyklických referencí. Při výskytu cyklických referencí by se měl použít i \type{std::weak\_ptr}.
}

Díky automatické správě paměti je vývoj výrazně rychlejší.
Automatická správa paměti ovšem neznamená, že program je chráněn proti všem únikům paměti.\footnote{V anj. \emph{memory leaks}.}
Pokud se např. \uv{opomene} odstranit záznam z mapy, odkazovaná instance nikdy nebude uvažována pro GC a vznikne únik paměti.

\subsubsection{Jak funguje GC}
Ve většině JVM se používá generační GC, který vychází ze dvou vypozorovaných chování:
\begin{itemize}
  \item Většina alokovaných objektů je velmi brzy nepotřebná, tzn. nejsou na ně žádná reference
  \item Jen několik starších objektů odkazuje na mladší objekty
\end{itemize}
Tyto dvě pozorování jsou známy jako \emph{slabá generační hypotéza}\cite{hunt2011java},\footnote{V anj. \emph{weak generational hypothesis}.} která platí pro většinu programů v Java.
K využití této hypotézy je v JVM rozdělena \emph{halda}\footnote{V anj. \emph{heap}} na dvě fyzické části, které jsou odkazované jako generace:

\missingfigure{Jak vypada alokace a promovani}
\begin{description}
  \item[Mladá generace] Většina objektů je alokována do této oblasti, která je relativně malá a je často \uv{uklízena}.
Jelikož je předpoklad, že objekty jsou velmi rychle nepotřebné, jen velmi málo objektů \uv{přežije} úklid v této oblasti.
Obecně je úklid v této oblasti velmi efektivní, protože je prováděn v malé oblasti paměti, která obsahuje hodně odpadních objektů.

  \item[Stará generace] Objekty, které \uv{žijí} déle, jsou promovány do této oblasti.
Tato oblast je výrazně větší než \emph{mladá generace} a její obsazování probíhá pomalu.
Úklid v této oblasti (nazývaný \emph{full GC}) není častý, ale pokud už k němu dojde, může být velmi dlouhý.

  \item[Permanentní generace] Toto je třetí oblastí generací, ale není použita v generační hierarchii.
JVM ji používá pro ukládání metadat jako jsou informace o třídách, internovaných řetězců atd\ldots
\end{description}

\subsubsection{Typy GC\label{subsub:gctypes}}

V Oracle JVM je k dispozici několik GC. Liší se způsobem provádění úklidu a kdy dochází k tzv. \emph{zastavení světa}.\footnote{V anj. \emph{stop the world}}
\emph{Zastavení světa} znamená, že všechna vlákna programu jsou pozastavena a pouze vlákna pro úklid jsou v činnosti.
Je to z důvodu toho, aby nedocházelo k novým alokacím objektů.
Samozřejmě, že toto má negativní dopad na aplikaci, pokud k zastavení dochází často.

\prettyref{tab:gcs} obsahuje porovnání GC. Tabulka vyjadřuje, jakým způsobem probíhá úklid paměti jednotlivých generací.

\begin{table}
\center
\begin{tabular}{|l || l | l | p{2cm} | p{2cm} |}
\hline
& Serial GC & Parallel GC & CMS GC & G1 GC \\ \hline \hline
Paralelismus & Ne & Ano & Ano & Ano \\ \hline
Souběžnost & Ne & Ne & Ano & Ano \\ \hline
Úklid mladé gen. & Sériově & Paralelně & Paralelně & Paralelně \\ \hline
Úklid staré gen. & Sériově & Paralelně & Paralelně \& \mbox{Souběžně} & Paralelně \& \mbox{Souběžně} \\
 \hline
\end{tabular}
\caption{Porovnání GC\label{tab:gcs}}
\end{table}

\subsection{Generika\label{sub:generika}}

Obdobně jako v~\CC{} jsou \emph{templates}\cite{ISO:2012:CPP}, jazyk Java má \emph{generics}.\footnote{Uvedeny ve verzi 1.5 \cite{gosling2013java}.}
Jsou zde rozdíly:
\begin{itemize}
  \item v~Java se informace o~typu ztrácí během překladu. Překladač nahrazuje výskyt generika za konkrétní typ (třída, rozhraní)
  \item nelze jako typový parametr použít primitivní typ -- \type{int, char, byte, long, double, float} \ldots{}
  \item nelze použít netypový parametr
\end{itemize}

Zřejmě největším přínosem je typová bezpečnost při práci s~kolekcemi a možnosti typové specializace.

\subsection{Překlad \bytecode{} do nativních instrukcí procesoru}

JVM je heap procesor. Kvůli emulaci a virtualizaci procesoru je
běh samotného \bytecode{} relativně pomalý. JVM proto překládá
za běhu \bytecode{} do nativních instrukcí procesoru, nad kterým
aktuálně JVM běží. V~JVM od Oracle
 se nazývá JIT\nomenclature{JIT}{Just In Time} Compiler\cite{hunt2011java}. Samozřejmostí je optimalizace při několika průchodech
kódem -- např. \emph{function inlining}, \emph{loop unrolling} nebo \emph{dead code elimination}\cite{hunt2011java}. Toto jsou velmi důležité vlastnosti, které mohou zásadním způsobem
ovlivnit měření výkonnosti programu. Více viz.~\prettyref{sec:vykonvjava}.

\subsection{Bohatá knihovna tříd v~základním JDK}

Java je dodávána s~rozsáhlou knihovnou JDK. Obsahuje vše nutné
pro síťovou komunikaci (RPC\nomenclature{RPC}{Remote Procedure Call}), práci s~textem -- regulární výrazy,
XML, soubory a souborový
systém a také velmi propracovanou knihovnou kontejnerů -- \emph{Java
Collections Framework }(JFC\nomenclature{JFC}{Java Collections Framework})
\footnote{V~jazyce Java se používá pro termín \emph{container} (\CC) termín
\emph{collection}}.

\section{Návrh a implementace}
Implementace \MIndex u je napsána pro verzi Java 7.
V~celém návrhu jsem se snažil co nejvíce používat objektově orientovaný princip.
Tzn. že jsem se maximálně snažil omezit používaní nižších konstrukcí jazyka Java jako jsou pole a spíše využívat knihovny JDK a vyšších konstrukcí jako jsou seznamy, fronty.
Všechny implementované třídy a rozhraní jsou typově parametrizované viz. \prettyref{sub:generika}.

Při některých výpočtech bylo možné použít \emph{rozděl a panuj}\cite{Cormen:2001:IA:580470}\footnote{V anj. \emph{Divide-and-Conquer}} s~využitím \emph{Executor Framework} z~JDK.

V~popisovaných algoritmech (ať už \BPTree{} nebo \MIndex) je často uváděna proměnná \type{leaf}, resp. zda je daná struktura list -- koncový uzel.
Z~hlediska objektového návrhu se mění chování třídy podle její datové složky, což nesvědčí o~dobrém objektovém návrhu.
K~odstranění tohoto jsem použil \emph{Replace Type Code with Subclasses}\cite{fowler1999refactoring} pro \BPTree.
V~případě \MIndex{} \emph{Range Query} jsem použil \emph{Visitor pattern}\cite{gamma1995design}.

\subsection{\BPTree}
Implementace \BPTree{} se nachází v~třídě \linebreak \type{cz.rank.vsfs.btree.BPlusTreeMultiMap}. Tato mapa je typově parametrizována pro klíč a hodnotu. Protože strom musí být setříděný, musí klíč implementovat rozhraní \type{java.lang.Comparable}. Mapa umožňuje ukládat více objektů pod stejným klíčem\footnote{V \CC{} je obdobný kontejner \type{std::multimap}\cite{ISO:2012:CPP}. V~JDK žádná takováto kolekce není.} a umožňuje vyhledání v~intervalu.

Implementace mazání objektů z~mapy není potřebná pro \MIndex{} a není tudíž ani implementována.

Celý \BPTree{} je uložen v~hlavní paměti a není tedy implementováno načítaní a ukládání uzlů na disk.

\subsection{\MIndex}
Celá implementace \MIndex{} pracuje s~objekty, které implementují rozhraní \type{cz.rank.vsfs.mindex.Distanceable}. Tím je zajištěna případná znovupoužitelnost kódu pro jiné datové typy než je vektor.

\subsubsection{Výpočet maximální vzdálenosti mezi prvky}
Výpočet maximální vzdálenosti je NP problém, kdy se musí spočítat vzájemné vzdálenosti všech prvků v~množině a vybrat tu vzdálenost, která je největší. K~výpočtu lze použít \emph{rozděl a panuj}, takže je celý výpočet paralelní a doba výpočtu se lineárně snižuje s~počtem výpočetních vláken.

Funkcionalita je implementována ve třídě \linebreak \type{cz.rank.vsfs.mindex.MaximumDistance}.
Vzdálenost může být spočítána pouze nad objekty, které implementují rozhraní \type{Distanceable}.


\subsubsection{Výpočet vzdáleností mezi pivoty a prvky}
K~výpočtu a seřazení vzdáleností pivotů od objektu slouží třídy implementující rozhraní \type{cz.rank.vsfs.mindex.PivotDistanceTable}. Implementovány jsou tři třídy s~tímto rozhraním.

\begin{figure}
\centering
\begin{mpost}[use,mpsettings={input metauml;}]

Interface.A("PivotDistanceTable")
	("+calculate()",
	 "+pivotAt(object:D, index:int):Pivot<D>",
	 "+firstPivotDistance(object:D):double",
	 "+distanceAt(object:D, index:int):double",
	 "+pivotDistance(object:D, pivotIndex:int):double",
);

ClassTemplate.TA("Distanceable<D>")(A);

AbstractClass.B("AbstractPivotDistanceTable")
	()();
ClassTemplate.TB("Distanceable<D>")(B);
Class.C("ParallelPivotDistanceTable")
	()();
ClassTemplate.TC("Distanceable<D>")(C);

Class.D("SimplePivotDistanceTable")
	()();
ClassTemplate.TD("Distanceable<D>")(D);

Group.g(C,D);

topToBottom.midx(30)(A,B,g);
leftToRight(80)(C,D);

drawObjects(A, B, g, TA, TB, TC, TD);
clink(realization)(B,A);
link(inheritance)(pathStepY(C.n,B.s,10));
link(inheritance)(pathStepY(D.n,B.s,10));

\end{mpost}

\caption{\type{PivotDistanceTable} UML diagram}
\end{figure}

\begin{description}
\item[ParallelPivotDistanceTable] Rozdělí úlohu na menší části podle objektů pomocí \emph{rozděl a panuj} a výpočet je paralelizován za použití všech dostupných jader systému. Tato třída je využita při konstrukci samotného \MIndex.
\item[SimplePivotDistanceTable] Podobně jako \type{ParallelPivotDistanceTable} vypočte vzdálenosti objektu od pivotů, ale pouze pro jeden objekt a výpočet není paralelní. Tato třída se používá v~samotném \emph{Range Query} dotazu. Pokud by se použil \type{ParallelPivotDistanceTable}, tak režije spojená s~vytvářením úloh pro exekutory, vytvořením vláken, by byla výrazně vyšší než samotný výpočet vzdáleností jednoho objektu.
\item[AbstractPivotDistanceTable] Je společným předkem obou výše uvedených tříd a obsahuje společný kód, který potomci využívají.
\end{description}

\subsubsection{Cluster tree}
\begin{figure}
\centering
\begin{mpost}[use,mpsettings={input metauml;}]

Interface.Cluster("Cluster")
	("+getIndex():Index",
	 "+accept(visitor:ClusterVisitor<D>)",
	 "+getLevel():int",
	 "+getSubCluster(pivot:Pivot<D>):Cluster<D>",
);

ClassTemplate.TCluster("Distanceable<D>")(Cluster);

Class.InternalCluster("InternalCluster")
	()();
ClassTemplate.TInternalCluster("Distanceable<D>")(InternalCluster);
Class.RootCluster("RootCluster")
	()();
ClassTemplate.TRootCluster("Distanceable<D>")(RootCluster);

Class.LeafCluster("LeafCluster")
	()();
ClassTemplate.TLeafCluster("Distanceable<D>")(LeafCluster);

Group.g(RootCluster,LeafCluster);

topToBottom.midx(30)(Cluster,InternalCluster,g);
leftToRight(80)(RootCluster,LeafCluster);

drawObjects(Cluster, InternalCluster, g, TCluster, TInternalCluster, TRootCluster, TLeafCluster);
clink(realization)(InternalCluster,Cluster);
link(inheritance)(pathStepY(RootCluster.n,InternalCluster.s,10));
link(inheritance)(pathStepY(LeafCluster.n,InternalCluster.s,10));

\end{mpost}

\caption{\type{Cluster} UML diagram}
\end{figure}

Cluster tree je tvořen třídami, které implementuji rozhraní \linebreak \type{cz.rank.vsfs.mindex.Cluster}. Jsou celkem 3 -- 
\type{InternalCluster} a jeho potomci  \type{LeafCluster} a \type{RootCluster}.

\begin{description}
\item[InternalCluster] Je použit jako vnitřní cluster. Obsahuje všechnu funkcionalitu pro udržení informací o~podstromech, objektech, informace o~minimální a maximální vzdálenosti uložené v~clusteru.
\item[LeafCluster] Reprezentuje koncový cluster. Jeho metoda \linebreak \method{LeafCluster.getSubCluster} vrací vždy konstantu \type{NO\_SUBCLUSTERS}. Toto slouží k~identifikaci koncového clusteru při utváření dynamického \MIndex u.
\item[RootCluster] Je použit pro kořen celého clusteru, protože vyžaduje speciální zacházení s~indexem a také při akceptování \type{ClusterVisitor}.
\end{description}

Pro změření rozdílu použití více-stupňového \MIndex u a \MIndex u s~dynamickými stupni jsou implementovány dvě třídy -- \linebreak \type{cz.rank.vsfs.mindex.MultiLevelClusterTreeBuilder} a \linebreak \type{cz.rank.vsfs.mindex.DynamicClusterTreeBuilder}. Tyto třídy jsou použity v~\type{cz.rank.vsfs.mindex.MultiLevelMIndex}, \linebreak resp. \type{cz.rank.vsfs.mindex.DynamicMIndex} ke konstrukci specifického \MIndex u.
Obě třídy mají téměř totožný konstruktor, pouze u~\type{DynamicClusterTreeBuilder} je navíc parametr pro maximální zaplnění koncového clusteru. Více viz \prettyref{sec:Dynamic-Cluster-Tree}.

\subsubsection{Range Query}
Díky nezávislosti dotazu na typu clusteru (více-stupňový vs. s dynamickými stupni) je \emph{Range Query} je implementována v třídě \type{MIndex}, která je společný rodič pro \type{MultiLevelMIndex} a \type{DynamicMIndex}. V metodě rangeQuery není celá logika vyhledávání. V této metodě se inicializuje ClusterRangeQuery, která implementuje samotné vyhledávání. Toto řešení jsem zvolil kvůli oddělení zodpovědnosti vyhledávání a uchovávání informací o \MIndex u.\footnote{Jedná se o tzv. \emph{Single Responsibility Principle\nomenclature{SRO}{Single Responsibility Principle}}\cite{martin2003agile}}

V podstatě je téměř doslovně přepsán algoritmus popsaný v \prettyref{sec:Range-Query}. Jak bylo již zmíněno v začátku této sekce, k odstranění nutnosti identifikování typu clusteru za běhu\footnote{V jazyce Java k identifikaci typu slouží operátor \method{instanceof}\cite{gosling2013java}, obdobně jako v \CC{} je operátor \method{typeid}\cite{ISO:2012:CPP}} je použit \emph{Visitor Pattern}. \prettyref{fig:rangequery-visitor-seq} popisuje toto volání detailně.


%\begin{figure}
%\centering
%\begin{sequencediagram}
%  \newthread{mi}{:MIndex}{MIndex}
%  \newinst{rq}{:ClusterRangeQuery}{}
%  \newinst{ic}{:InternalCluster}{}
%  \newinst{lc}{:LeafCluster}{}
%
%  \begin{call}{mi}{performQuery()}{rq}{nalezené objekty}
%    \begin{sdblock}{Dotaz}{dokud není fronta clusterů prázdná}
%    \end{sdblock}
%  \end{call}
%\end{sequencediagram}
%
%\caption{Sekvenční diagram volání \emph{Range Query}}\label{fig:rangequery-visitor-seq}
%\end{figure}
\missingfigure{Sekvenční volání RangeQuery}

\section{Měření výkonu v~Java\label{sec:vykonvjava}}

Abychom porozuměli úskalí měření výkonu kódu v~JVM, je nejprve nutné si přiblížit, co se děje uvnitř JVM, jak probíhá překlad a optimalizace kódu.

\subsection{Zavádění tříd}

Všechny třídy jsou přeloženy do \classfile{} souborů. Ty mohou být následně zabaleny do JAR\nomenclature{JAR}{Java ARchive}, což je v~podstatě ZIP archív. JVM při startu vyhledá tzv. Main-Class a k~ní všechny závislé třídy a k~nim další závislé třídy v~tzv. \emph{classpath}. U~každé třídy se provede statická inicializace. Toto vyhledávání není omezeno pouze na úvodní spuštění JVM. Volání \method{Class.forName()}, \method{ClassLoader.loadClass()}, \emph{Reflection API} a \method{JNI\_FindClass} může vyvolat \emph{zavádění tříd}\footnote{V anj. \emph{class loading}} kdykoliv za běhu programu, pokud požadovaná třída není již nahrána\cite{gosling2013java}\cite{lindholm2013java}.

Další fází při \emph{zavádění tříd} je i verifikace \bytecode{}. Všechny překladače Java (\javac{}) vytvářejí validní \classfile{} soubory a typově bezpečný kód. Jenže JVM se nemůže spolehnout, že soubor, který nahrála je správný. Proto musí ověřit, že je vše v~pořádku -- ať již instrukce zadané v~\classfile{} souboru nebo typová bezpečnost.

Výše popsaný proces \emph{zavádění tříd} je velmi náročný na I/O operace a zásadním způsobem ovlivní výsledek měření, pokud se nepředejde tomuto zavádění během měření.

\subsection{JIT\label{sub:JIT}}
Tradiční překladače (např. C/\CC) generují z~vyššího programovacího jazyka strojově závislý kód. Tento kód má finální podobu a všechny možné optimalizace je zapotřebí udělat během překladu. Případně použít profilování výsledné binárky a znovu nechat překladač znovu přeložit kód s~využitím profilovacích dat. Těmto překladačům se říká také statické. Výsledný kód je optimalizován pro danou platformu.

Java používá překladač \javac{}, který z~vysoko úrovňového jazyka vytvoří \classfile{}, který obsahuje \bytecode{}. JVM tento \bytecode{} následně dynamicky za běhu překládá do strojově závislého kódu pomocí JIT.

\subsubsection{Jak funguje překladač}
Každý překladač má podobnou strukturu.
Musí mít na vstupu modul pro převod zdrojového kódu do tzv. \emph{intermediate representation (IR)}\nomenclature{IR}{Intermediate Representation}.
IR je reprezentace programovacího jazyka do vnitřní formy překladače.
Existují různé typy IR. Pro každou fázi překladu se dokonce může použít jiný IR.
Všechny překladače dělají největší optimalizace právě na IR.
Množina možných optimalizací může být velmi široká a je často omezena množstvím času nutném k~provedení dané optimalizace.
Mezi jednoduché optimalizace patří zjednodušování logických výrazů, nahrazování proměnných konstantami a \emph{function inlining}.
Složitější optimalizace jsou většinou spojeny se smyčkami -- loop unrolling, odstraňování kontroly rozsahu smyčky atd\ldots{}

Když jsou tyto optimalizace hotovy, další modul vezme danou IR a převede ji do strojové podoby. Zde nastávají další optimalizace v~podobě přiřazování hodnot do registrů procesoru, výběr vhodných instrukcí atd\ldots{}

\subsubsection{Kdy JIT překládá do strojového kódu\label{subsub:whenJIT}}
JIT nepřekládá každou instrukci v~\bytecode{} ihned do strojového kódu. Většina instrukcí je na začátku interpretována. Až v~okamžiku, kdy daná instrukce se stane \emph{hot}, JIT ji zařadí pro překlad. JVM totiž udržuje u~každé metody čítač volání. Pokud daná metoda je volána více krát, stane se \emph{hot} a tím se vyvolá i její překlad. Hranice překladu je ve výchozím nastavení 10000.\footnote{Parametr pro Oracle JVM, který ovlivňuje toto nastavení je \cmd{-XX:CompileThreshold=}} Zjednodušeně řečeno: aby došlo k~překladu libovolné metody do strojového kódu, musí být daná metoda volána aspoň 10000.

Níže jsou uvedeny optimalizace, které ovlivňují výkonnost kódu.

\begin{description}
\item[function inlining] Je technika vnořování těla kódu funkcí/metod do těla volané metody. Protože každá metoda třídy je v jazyce Java virtuální, tato optimalizace odstraňuje penalizaci za jejich volání. Ve výsledku vytváří velké bloky kódu, nad kterými se uplatňují další optimalizace. Vnořování nedochází u všech metod. Podmínkou je, aby metoda byla krátká, což znamená, aby měla velikost do 35 \bytecode s.\footnote{V Oracle JVM lze toto omezení upravit pomocí parametru \cmd{-XX:FreqInlineSize=}} Vnořování nemůže být vynuceno. Záleží na JIT. Není ani žádné doporučení pro kompilátor jako je tomu u \CC{} pomocí vyhrazeného slova \method{inline} u metody.\footnote{\method{inline} říká kompilátoru, že by mohl metodu vnořit, ale záleží na chytrosti kompilátoru, zda vnoření provede\cite{Stroustrup:1997:CPL:523265}.}

Tato technika má zásadní dopad na běh programu. Ve výsledku nutí programátora navrhnout třídy tak, aby byly tvořeny malými metodami, což má za následek i dobrou čitelnost kódu.

\item[dead code elimination] JIT je schopen při analýze běhu programu identifikovat části kódu, které nemají žádný vedlejší efekt, např. pokud se výsledek volání metody nepoužije v dalších krocích nebo. Pokud je měřící kód nesprávně navrhnut, může dojít k tomu, že díky této optimalizaci se některé části nevykonají a výsledky měření jsou nepoužitelné.

\item[loop unrolling] Slouží k rozbalení smyček jako je \method{for (...) ...}. Klasická optimalizace, která tělo smyčky \uv{zkopíruje} několikrát za sebou. Každá smyčka má inkrementální část a podmínku pro ukončení smyčky. Podmínka pro ukončení má vliv na \emph{branch prediction} v CPU.\footnote{Moderní CPU mají velmi sofistikované algoritmy na předpovídání větvení programu s předčítáním instrukcí do \emph{pipeline}. Pokud CPU špatně odhadne následující větev kódu, v drtivé většině případů musí \uv{zahodit} a znovu načíst instrukce do \emph{pipeline}. Tato operace je velmi drahá a zastaví na několik cyklů vykonávání instrukcí CPU.} Ve výsledku se daná podmínka nevyhodnocuje tak často.

\begin{figure}
\begin{lstlisting}[frame=single,language=Java,caption=Před loop unrolling]
for (int x = 0; x < 50; x++) {
  doSomethingWith(x);
}
\end{lstlisting}
\begin{lstlisting}[frame=single,language=Java,caption=Po loop unrolling]
for (int x = 0; x < 50; x+=5) {
  doSomethingWith(x);
  doSomethingWith(x+1);
  doSomethingWith(x+2);
  doSomethingWith(x+3);
  doSomethingWith(x+4);
}
\end{lstlisting}
\end{figure}
\end{description}

\subsection{Zahřátí JVM}

Pokud bychom začali měřit jednotlivé části kódu ihned po startu JVM, neměřili bychom výkonnost samotného kódu, ale i čas za jak dlouho JVM zavede jednotlivé \classfile soubory, za jak dlouho JIT přeloží a optimalizuje instrukce \bytecode{} do strojového kódu. Pravděpodobně by se i naměřené hodnoty pro stejné testované parametry lišily, protože by k~překladu docházelo v~průběhu nebo mezi jednotlivými měřeními. Proto je nutné udělat i tzv. \emph{zahřátí JVM}.\footnote{V anj. \emph{JVM warm-up}}

Zahřátí se provádí tak, že se zavolá měřený kód tolikrát, aby se stal \emph{hot}.\footnote{Viz \prettyref{subsub:whenJIT}} Po tomto zahřátí, by již měl být měřený kód optimalizován a převeden do strojového kódu a měření by nemělo být ovlivněno.

\subsection{GC během měření}
Jak bylo uvedeno v \prettyref{subsub:gctypes}, během úklidu paměti může dojít k zastavení aplikace.
Je otázka, jak tomuto zastavení přistupovat při měření.
V reálné aplikaci je toto běžné a záleží na typu aplikace, zda je zastavení celé aplikace na určitý časový okamžik problém nebo ho lze tolerovat.
Podle typu aplikace je tedy vhodné vybrat i správný \emph{garbage collector}.

V případě testování výkonnosti určitého algoritmu je zřejmě vhodné zabránit úklidu během měření, protože ve výsledném měření bude i započítán čas nutný pro úklid paměti.
Z aplikace je možné vynutit úklid pomocí voláni \method{System.gc()} a \uv{počkat}, dokud neproběhne úklid paměti.
V měřeních jsem tohoto volání využil, aby se opravdu měřil čas nutný k vykonání \emph{Range Query}.

\section{Metoda a výsledky měření}
\subsection{Metoda měření\label{sec:measurementmethod}}

\MIndex{} má několik parametrů, které ovlivňují jeho chování, resp. výkonnost, v závislosti na velikosti a typu indexovaných dat. Parametry pro měření jsou následující:
\begin{description}
\item[počet pivotů] z kolika pivotů má být vytvořen \MIndex{}
\item[počet vyhl. objektů] kolik se má vyhledat objektů v rámci jednoho měření
\item[max. úroveň clusteru] platí pro oba dva typy clusterů
\item[počet měření] kolik měření se má provést s danými parametry
\item[max. počet objektů v listech] tento parametr je pouze při měřeních dynamického clusteru a určuje maximum objektů v listu clusteru
\item[vzdálenost] maximální vzdálenost hledaných objektů od vzorového
\item[stupeň \BPTree{}] určuje počet klíčů v uzlech \BPTree{}. Jelikož není implementováno načítání a ukládání uzlů na disk, tento parametr ovlivňuje samotné vyhledávání ve stromu
\end{description}

Výkonnost \MIndex u ovlivňuje i nastavení parametrů JVM.
Zde jsem se rozhodl prozkoumat 2 parametry -- \cmd{-XX:+AggressiveOpts} a \cmd{-XX:+UseCompressedOops}, tzn. jejich vliv na výsledné měření a na jejich kombinace.

\begin{description}
\item[AggressiveOpts] Tato volba povoluje agresivní optimalizace.
Je označena jako experimentální a rozsah optimalizací se může být jiná v každé verzi JVM.
\item[UseCompressedOops] Tato volba je dostupná pouze pro 64 bitovou verzi JVM.
V JVM jsou všechny ukazatele na objekty schodné se strojovými ukazateli, tzn. na 64 bitové architektuře jsou 64 bitové, na 32 bitové architektuře jsou 32 bitové.
Obecně je používání 64 bitových instrukcí pomalejší oproti ekvivalentním 32 bitovým instrukcím -- vyžadují i odpovídající registry atd\ldots{}
Při zapnuté volbě \emph{UseCompressedOops} JVM komprimuje ukazatele na 32 bitů a umožňuje používat 64 bitové registry pro jiné operace, čímž snižuje penalizaci za 64 bitové instrukce.\footnote{Toto je velmi jednoduché přiblížení. Více viz \cite{hunt2011java}}
\end{description}


Průběh testu je popsán níže.

\begin{enumerate}
\item Vygenerování parametrů pro měření.
\item Načtení vstupních dat ze souboru a vytvoření vektorů.
\item Výpočet maximální vzdálenosti mezi objekty mimo třídy \type{*MIndex}, aby se pro každé nové měření nemusela znovu počítat vzdálenost.
Každá třída \type{cz.rank.vsfs.mindex.MultiLevelMIndex} a \type{cz.rank.vsfs.mindex.DynamiclMIndex} má konstruktor, který má parametr pro předpočítanou maximální vzdálenost.
\item Zahřátí JVM.
\item Opakující se test:
\begin{enumerate}
\item Vytvoření pivotů.
\item Vytvoření instance \type{\MIndex}.
\item Přiřazení vektorů do instance \type{\MIndex}.
\item Vytvoření vnitřní struktury \MIndex u voláním metody \method{MIndex.build()}.
\item Spuštění časovače.
\item Jednotlivé dotazy na podobnost voláním \method{MIndex.rangeQuery()}
\item Vypnutí časovače.
\item Vypsání informací o měření -- parametry a čas vyhledávání
\item Spuštění úklidu paměti.
\end{enumerate}

Výsledné časy byly zprůměrovány.
Časovač zaznamená čas zapnutí a čas vypnutí pomocí \method{System.currentTimeInMillis()} a rozdíl je čas měření v milisekundách.
Při tomto způsobu se nezaznamenává pouze čas vlákna, kolik taktů bylo aktivní, ale celková doba, tzn. pokud je systém vytížen a měřené aplikaci se nedostává času procesoru, je měření tímto postiženo.
Během měření žádné další uživatelské procesy neběžely, aby nedošlo k ovlivnění měření.

\end{enumerate}
\subsection{Prostředí}
Měření probíhalo v tomto prostředí:

\begin{table}[h]
\catcode`\-=12
\center
\begin{tabular}{|l  | l | c |}
\hline
\multirow{9}{*}{CPU} & \multicolumn{2}{|c|}{AMD Phenom\texttrademark II X4 905e Processor} \\ \cline{2-3}
& Architektura & 64bit \\ \cline{2-3}
& Počet jader & 4 \\ \cline{2-3}
& Počet vláken & 4 \\ \cline{2-3}
& Frekvence & 2.5Ghz \\ \cline{2-3}
&  \multirow{2}{*}{L1 cache} & 4 x 64 KB pro instrukce \\
& & 4 x 64 KB pro data \\ \cline{2-3}
& L2 cache & 4 x 512KB \\ \cline{2-3}
& L3 cache & 6MB \\ \cline{2-3}
\hline
\multirow{3}{*}{RAM} & Velikost & 12GB \\ \cline{2-3}
& Typ & DDR3 \\ \cline{2-3}
& Frekvence & 1333MHz \\ \cline{2-3}
\hline
\multirow{2}{*}{Operační systém} &  \multicolumn{2}{|c|}{Debian/GNU Linux 7.0}  \\
&  \multicolumn{2}{|c|}{\tiny 3.8-trunk-amd64 \#1 SMP Debian 3.8.3-1~experimental.1 x86\_64 GNU/Linux} \\ \cline{2-3}
\hline
\multirow{2}{*}{Java} & JDK & {\tiny Java(TM) SE Runtime Environment (build 1.7.0\_13-b20)} \\ \cline{2-3}
& JVM & {\tiny Java HotSpot(TM) 64-Bit Server VM (build 23.7-b01, mixed mode)} \\ \cline{2-3}
 \hline
\end{tabular}
\caption{Informace o stroji}
\end{table}

\subsection{Parametry měření}
\begin{table}[h]
\center
\begin{tabular}{|l|c|}
\hline
Parametr & Hodnot(a/y) \\ \hline
\hline
počet pivotů & 10, 20, 30, 50 \\ \hline
počet vyhl. objektů & 100 \\ \hline
max. úroveň clusteru & 2, 3, 4 \\ \hline
počet měření & 20 \\ \hline
max. počet objektů v listech & 10, 20, 50, 100, 200, 300, 500 \\ \hline
vzdálenost & 0.15 \\ \hline
řád \BPTree{} & 50, 100, 500 \\ \hline
\end{tabular}
\end{table}

Původně jsem chtěl změřit více hodnot, např. počet pivotů 100 s max. počtem objektů v listech 10 a úrovní clusteru 5.
Při těchto parametrech ovšem velikost clusterů byla extrémně velká -- $100 \times 99 \times 98 \times 97 \times 96 = 9034502400$.
Při práci s tímto množstvím docházelo k přetečení typu \type{Integer} v kolekcích a nesmyslným výsledkům.
\footnote{V Java je typ \type{Integer} 4 bajtový celočíselný typ se znaménkem a může mít maximálně hodnotu $2^{31}-1$}

Při tomto množství je celkový počet kombinací parametrů 252 pro dynamický cluster a 36 pro více-stupňový cluster.
Jelikož počet měření je 20 pro každý parametr, bylo celkově vykonáno 5760 měření.

Každá série měření byla spouštěna s JVM s různými kombinacemi parametrů, které byly uvedeny v \prettyref{sec:measurementmethod}.
Pro další výklad jsem se rozhodl použít následující označení:

\begin{table}[h]
\center
\begin{tabular}{|l  || c | c |}
\hline
& AggresiveOpts & CompressedOops \\ \hline
\hline
JVM1 & Zapnuto & Zapnuto \\ \hline
JVM2 & Vypnuto & Zapnuto \\ \hline
JVM3 & Zapnuto & Vypnuto \\ \hline
JVM4 & Vypnuto & Vypnuto \\ \hline
\end{tabular}
\caption{Označení kombinací parametrů JVM}
\end{table}

\subsection{Výsledky měření}
Grafy z měření se nacházejí jako \prettyref{ape:graphs}. Tabulkový přehled výsledků měření jsou viz \prettyref{ape:tables}.

\section{Srovnání výkonu implementací v~{\protect \CC}, C\# a Java}
\subsection{Srovnání implementací}
\todo{Porovnat implementaci p. Kmnuníčka s~mojí}
\subsection{Srovnání výsledků měření}
\todo{Vybrat stejné parametru měření a porovnat je navzájem}
